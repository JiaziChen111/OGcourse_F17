\documentclass[letterpaper,12pt]{article}
\usepackage{array}
\usepackage{threeparttable}
\usepackage{geometry}
\geometry{letterpaper,tmargin=1in,bmargin=1in,lmargin=1.25in,rmargin=1.25in}
\usepackage{fancyhdr,lastpage}
\pagestyle{fancy}
\lhead{}
\chead{}
\rhead{}
\lfoot{}
\cfoot{}
\rfoot{\footnotesize\textsl{Page \thepage\ of \pageref{LastPage}}}
\renewcommand\headrulewidth{0pt}
\renewcommand\footrulewidth{0pt}
\usepackage[format=hang,font=normalsize,labelfont=bf]{caption}
\usepackage{listings}
\lstset{frame=single,
  language=Python,
  showstringspaces=false,
  columns=flexible,
  basicstyle={\small\ttfamily},
  numbers=none,
  breaklines=true,
  breakatwhitespace=true
  tabsize=3
}
\usepackage{amsmath}
\usepackage{amssymb}
\usepackage{amsthm}
\usepackage{harvard}
\usepackage{setspace}
\usepackage{float,color}
\usepackage[pdftex]{graphicx}
\usepackage{hyperref}
\hypersetup{colorlinks,linkcolor=red,urlcolor=blue}
\theoremstyle{definition}
\newtheorem{theorem}{Theorem}
\newtheorem{acknowledgement}[theorem]{Acknowledgement}
\newtheorem{algorithm}[theorem]{Algorithm}
\newtheorem{axiom}[theorem]{Axiom}
\newtheorem{case}[theorem]{Case}
\newtheorem{claim}[theorem]{Claim}
\newtheorem{conclusion}[theorem]{Conclusion}
\newtheorem{condition}[theorem]{Condition}
\newtheorem{conjecture}[theorem]{Conjecture}
\newtheorem{corollary}[theorem]{Corollary}
\newtheorem{criterion}[theorem]{Criterion}
\newtheorem{definition}[theorem]{Definition}
\newtheorem{derivation}{Derivation} % Number derivations on their own
\newtheorem{example}[theorem]{Example}
\newtheorem{exercise}[theorem]{Exercise}
\newtheorem{lemma}[theorem]{Lemma}
\newtheorem{notation}[theorem]{Notation}
\newtheorem{problem}[theorem]{Problem}
\newtheorem{proposition}{Proposition} % Number propositions on their own
\newtheorem{remark}[theorem]{Remark}
\newtheorem{solution}[theorem]{Solution}
\newtheorem{summary}[theorem]{Summary}
%\numberwithin{equation}{section}
\bibliographystyle{aer}
\newcommand\ve{\varepsilon}
\newcommand\boldline{\arrayrulewidth{1pt}\hline}


\begin{document}

\begin{flushleft}
  \textbf{\large{Problem Set \#1}} \\
  MACS 40000, Dr. Evans \\
  Fiona Fan 
\end{flushleft}

\vspace{5mm}

\noindent\textbf{Problem 1}

\textbf{Part (a).} The Lagrangian is:
\begin{equation*}
	\underset{c_{1,t}, c_{2,t+1}}{max}
  L= (1-\beta)ln(c_{1,t})+\beta ln(c_{2,t+1})-\lambda (p_t c_{1,t}+p_{t+1} c_{2,t+1}-p_t e_1-p_{t+1} e_2)
\end{equation*}

To satisfy first order condition:
\begin{equation*} 
\frac{\partial L}{\partial c_{1,t}} = \frac{1-\beta}{c_{1,t}}-\lambda p_t = 0
\end{equation*}
\begin{equation} \label{c1foc}
\Rightarrow \lambda = \frac{1-\beta}{c_{1,t}  p_t}
\end{equation}
\begin{equation*} 
\frac{\partial L}{\partial c_{2,t+1}} = \frac{\beta}{c_{2,t+1}}-\lambda p_{t+1} = 0
\end{equation*}
\begin{equation} \label{c2foc}
\Rightarrow \lambda = \frac{\beta}{c_{2,t+1} p_{t+1}}
\end{equation}

from \eqref{c1foc} and \eqref{c2foc} we can get the Euler equation:
\begin{equation} \label{euler}
c_{2,t+1}=\frac{\beta p_t}{(1-\beta) p_{t+1}} c_{1,t}
\end{equation}

To satisfy complementary slackness condition:
\begin{equation*}
\lambda (p_t c_{1,t}+p_{t+1} c_{2,t+1}-p_t e_1-p_{t+1} e_2) = 0
\end {equation*}
However, here $\lambda \neq 0$ because $\beta \neq 1$ and $c_{1,t} \neq 0$. Thus,
\begin{equation} \label{constraint}
p_t c_{1,t}+p_{t+1} c_{2,t+1}-p_t e_1-p_{t+1} e_2 = 0
\end {equation}
Substituting \eqref{euler} into \eqref{constraint}
\begin{equation} \label{c1a}
\Rightarrow c_{1,t}^*=\frac{(p_t e_1+p_{t+1} e_2)(1-\beta)}{p_t}
\end {equation}
\begin{equation} \label{c2a}
\Rightarrow c_{2,t+1}^*=\frac{(p_t e_1+p_{t+1} e_2) \beta}{p_{t+1}}
\end {equation}

\textbf{Part (b).} The Lagrangian is:
\begin{equation*}
	\underset{c_{2,1}}{max}
   L= \beta ln(c_{2,1})-\lambda (p_1 c_{2,1} - p_1 e_2)
\end{equation*}

To satisfy first order condition:
\begin{equation*} 
\frac{\partial L}{\partial c_{2,1}} = \frac{\beta}{c_{2,1}}-\lambda p1 = 0
\end{equation*}
To satisfy complementary slackness condition:
\begin{equation*}
\lambda (p_1 c_{2,1}-p_1 e_2) = 0
\end {equation*}
However, here $\lambda \neq 0$ because $\beta \neq 1$ and $c_{2,1} \neq 0$. Thus,
\begin{equation*} 
\Rightarrow c_{2,1}^*=\frac{p_1 e_2}{p_1} = e_2
\end{equation*}

\textbf{Part (c).}  Plugging our answer in part (b) into the CE budget constraint we have:

\begin{align*}
c_{1,1}^*+c_{2,1}^*=e_1+e_2\\
c_{1,1}^*+e_2=e_1+e_2\\
\Rightarrow c_{1,1}^*=e_1
\end{align*}
Here, both the young and old consume all their repective endowments. There is no inter-generational exchange happening. These answers are different from the answers in part (a).
\begin{equation*}
\{c_{1,t},c_{2,t}\}_{t=1}^\infty = \{e_1,e_2\}
\end{equation*}
Since there is no inter-generational exchange happening, the prices can be arbitrary. 

\end{document}
\begin{equation*}
\end {equation*}
